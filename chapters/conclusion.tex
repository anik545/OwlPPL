% !TeX root = ../diss.tex
\chapter{Conclusion}

In this project, I have designed, developed and tested a probabilistic programming language embedding in OCaml. It has achieved all the core requirements as well as some of the extensions. The performance is reasonable when compared to similar systems (other universal PPLs). Code written in my PPL is also not overly verbose compared to these languages.

Future work would mainly focus on how to improve inference. There are several algorithms I have implemented that would benefit from the use of multiple cores - which may be possible with the ongoing development of multi-core OCaml. 

Other, more efficient inference algorithms may also be able to be written. This may require changing the core data structure or adding more variants to give more information, for example adding variable names in order to keep track of the primitives being used, or allowing a user to specify guide distributions to create more specific proposal distributions for MCMC.

One of my initial goals was for my PPL to be able to represent as many distributions (models) as possible. This is why I used a trace based approach inspired by Church rather than using a computation graph. However, there are universal PPLs which use dynamic computation graphs to make inference more efficient (e.g. Pyro). Since \texttt{Owl} contains a powerful computational graph implementation, this could be a further extension. I was not able to complete one extension due to the fact that I had not chosen this approach - the ability to visualise the model itself (e.g. as a network). If I used a graph approach, this would likely be possible.

My visualisations, while basic, allow producing graphs of posterior distributions and exact distributions. Extensions include visualising higher-dimensional data.

While there are many possible extensions to this project, it successfully achieved all the initial goals.

\mbox{}
\vfill
[ {\small \textbf{Word count}: \input{wordcount.txt}}]